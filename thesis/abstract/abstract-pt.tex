A cada 23 segundos, uma pessoa morre nas estradas. Em 2018, 1.35 milhões de pessoas morreram devido a acidentes nas estradas, 90\% dos quais foram devidos a erro humano: condução perigosa, distrações, fadiga e más decisões. Veículos autónomos são uma das soluções apresentadas para resolver este problema, substituindo ou ajudando o condutor. Para tal, os veículos precisam de conseguir perceber aquilo que os rodeia com grande precisão, sendo o LiDAR um dos sensores mais promissores para essa tarefa.

Para compreender o que os rodeia, os LiDARs emitem raios laser que podem, teoricamente, ser recebidos por um outro LiDAR, noutro carro, interferindo com a capacidade desse segundo LiDAR compreender o que rodeia. Num cenário onde múltiplos carros autónomos equipados com LiDAR coexistem, a sua interferência mútua pode comprometer a sua capacidade para perceber o que o rodeia com precisão e a possibilidade de solucionar um dos problemas que inicialmente ira resolver: acidentes e mortes na estrada.

Nesta Dissertação de Mestrado, propomos o estudo do comportamento da interferência entre dois LiDARs em vários cenários de interferência, onde variamos a sua distância, altura e posição relativa. Tentámos também perceber o diferente impacto da interferência direta e dispersa, através da obstrução da linha de vista entre os dois LiDARs, e verificar qual o comportamento da interferência em regiões de interesse e objetos. Construímos um setup experimental contendo dois LiDARs e uma câmara, calibramo-los intrínseca e extrinsecamente e estimamos a posição dos objetos de interesse na \textit{point cloud} através de regiões de interesse previamente detetadas em imagem. Usando este setup experimental, recolhemos mais de $600$~GB de dados não tratados, aos quais aplicamos 4 técnicas de análise de interferência diferentes, todas desenvolvidas por nós.

As nossas descobertas permite afirmar que o número relativo de pontos com interferência variam entre as ordens de magnitude de $10^{-7}$ e $10^{-3}$. Os nossos resultados mostram que a interferência direta predomina sobre a interferência dispersiva, causando com que o valor da interferência relativa seja uma ordem de magnitude maior se a linha de vista entre os dois LiDARs for obstruída. Somos também capazes de identificar situações em que a interferência se comporta de forma parecida ao ruído do sensor, sendo quase indistinguível; e outros casos em que esta está fortemente presente, causando erros nas medições de distância que ultrapassam até as dimensões físicas do espaço onde o setup experimental está a ser operado.

Concluímos que a interferência não aparenta ser tão destrutiva para condução autónoma como inicialmente previsto, devido à baixa ordem de grandeza da magnitude. De qualquer forma, esta pode ainda ter efeitos graves, principalmente em situações de interferência direta. Podemos também concluir que a natureza da interferência é altamente volátil, dependendo de condições ainda não 100\% definidas, incluindo a influência como é criado o setup experimental.

%É desejável prosseguir com análise do \textit{dataset} recolhido para expandir as nossas descobertas. Contudo, mais testes noutras condições (e com outros modelos de LiDAR) são necessárias para uma compreensão detalhada do comportamento da interferência entre LiDARs. Simular o comportamento da interferência é mandatório num próximo passo.

