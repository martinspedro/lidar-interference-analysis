Every 23 seconds, someone dies on the road. In 2018, 1.35 million people died because of a road accident, 90\% of which were caused by human error: reckless behavior, distractions, fatigue, and bad decisions. Autonomous vehicles are one of the solutions to tackle this problem, by replacing or helping the human driver on the driving task. For that, vehicles need to understand the world around them with great precision, from here LiDAR appears as one of the most promising sensors up for the task. 

To sense their surroundings, LiDARs emit laser beams, which can, theoretically, be received by a LiDAR on another car, disturbing the accuracy of their ability to map their environment. In a scenario when multiple autonomous vehicles equipped with LiDAR coexist, their mutual interference can undermine their capability to accurately understand the world and their capability to tackle one of the problems they came to solve: road accidents. 

On this Master's thesis we propose to study the behavior of two LiDARs on several interference scenarios, varying their relative distance, height and positioning. We also attempt to understand the different impacts of direct and scattered interference, by blocking the LiDARs line of sight and verify the behavior of the interference on specific regions of interest and objects. We construct an experimental setup containing two LiDARs and a camera, intrinsically and extrinsically calibrate them and estimate the position of the objects of interest on the point cloud through regions of intereset previously detected on the image. Using this experimental setup we gathered more than $600$~GB of raw data on which we apply 4 different techniques of interference analysis, all developed by us.

Our findings show that the relative number of interference points range between an order of magnitude of $10^{-7}$ to $10^{-3}$. The results show also that direct interference predominates over scattered, generating relative values of interfered points one order of magnitude higher if the line of sight between the LiDARs is obstructed. We were able to identify cases on which interference seems to behave closely to the sensor noise, being almost indistinguishable; and other causes when it was strongly present, resulting on depth measurement errors that can surpass even the physical dimensions of the room where the setup is operating. 

We can conclude that the interference does not appear to be as destructive for autonomous driving as our initial thoughts, since the order of magnitude being low. Nevertheless, it can still have severe effects, principally on situations of direct interference. We also conclude that its nature is highly volatile, depending on conditions not yet fully understand, including the influence of the experimental setup.

The dataset gathered on this work is still not fully explored and further analysis is desirable to expand our findings. However, more tests on different conditions, with more LiDARs (and with different  models) are required to fully understand the behavior of LiDAR Interference. Simulating the behavior of the interference is also a mandatory next step.



