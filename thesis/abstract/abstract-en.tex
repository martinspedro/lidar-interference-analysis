Every 23 seconds, someone dies on the road.  On 2018, 1.35 million people died because of a road accident, 90\% of which were caused by human error: reckless behavior, distractions, fatigue, and bad decisions. Autonomous vehicles are presented as one of the solutions to tackle this problem, by replacing the human driver on the driving task. For that, vehicles need to understand the world around them with greater precision. LiDAR appears as one of the most promising  sensors to understand vehicles drive better and safer. 

To sense their surroundings, LiDARs emit laser beams, which can, theoretically, be received by a LiDAR on another car, disturbing the accuracy of their ability to map their environment. In a scenario when multiple autonomous vehicles equipped with LiDAR coexist, their mutual interference can undermine their capability to accurately understand the world and their capability to tackle one of the problems they came to solve: road accidents. 

On this Master's thesis we propose to study the behavior of two LiDARs on several interference scenarios, varying their relative distance, height and positioning. We also try to understand the different impacts of direct and scattered interference, by blocking the LiDARs line of sight. The experimental setup devised also contains a camera, allowing data fusion between the depth information of the LiDAR and the color of the camera. The fusion of this data allows us to analyze how the interference behaves for objects of interest, such as persons, balls and boxes; which are detected by the camera and their real position estimated for the point cloud using an algorithm developed.

Our findings indicate that the test performed are not sufficient and that the parameters varied are only a subset of the ones that affect the behavior of LiDAR interference. We observed relative values of interfered points to the total number of points ranging from one in one thousand points to one in a million points. We were able to identify cases on which interference seems to be behave closely to the sensor noise, being almost indistinguishable; and other causes when it was strongly present, resulting on depth errors that can surpass  even the physical dimensions of the room where the setup is operating. 

We can conclude that the interference does not appear to be as destructive for autonomous cars as our initial thoughts, since the order of magnitude are low, but still can have severe effects on some occasions. We also conclude that its nature is highly volatile, depending on constrains not yet fully understand. 

The large amount and variety of data gathered is still not fully explored and further analysis are possible to expand our findings. However, more tests on different conditions and with more LiDARs (and with different  models) are required to fully understand the behavior of LiDAR Interference. Simulating the behavior of the interference is also mandatory next step.




